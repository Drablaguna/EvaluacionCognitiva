\documentclass[]{article}
\usepackage{lmodern}
\usepackage{amssymb,amsmath}
\usepackage{ifxetex,ifluatex}
\usepackage{fixltx2e} % provides \textsubscript
\ifnum 0\ifxetex 1\fi\ifluatex 1\fi=0 % if pdftex
  \usepackage[T1]{fontenc}
  \usepackage[utf8]{inputenc}
\else % if luatex or xelatex
  \ifxetex
    \usepackage{mathspec}
  \else
    \usepackage{fontspec}
  \fi
  \defaultfontfeatures{Ligatures=TeX,Scale=MatchLowercase}
\fi
% use upquote if available, for straight quotes in verbatim environments
\IfFileExists{upquote.sty}{\usepackage{upquote}}{}
% use microtype if available
\IfFileExists{microtype.sty}{%
\usepackage{microtype}
\UseMicrotypeSet[protrusion]{basicmath} % disable protrusion for tt fonts
}{}
\usepackage[margin=1in]{geometry}
\usepackage{hyperref}
\hypersetup{unicode=true,
            pdftitle={Capitulo 2. Análisis de Sentimientos},
            pdfborder={0 0 0},
            breaklinks=true}
\urlstyle{same}  % don't use monospace font for urls
\usepackage{color}
\usepackage{fancyvrb}
\newcommand{\VerbBar}{|}
\newcommand{\VERB}{\Verb[commandchars=\\\{\}]}
\DefineVerbatimEnvironment{Highlighting}{Verbatim}{commandchars=\\\{\}}
% Add ',fontsize=\small' for more characters per line
\usepackage{framed}
\definecolor{shadecolor}{RGB}{248,248,248}
\newenvironment{Shaded}{\begin{snugshade}}{\end{snugshade}}
\newcommand{\KeywordTok}[1]{\textcolor[rgb]{0.13,0.29,0.53}{\textbf{#1}}}
\newcommand{\DataTypeTok}[1]{\textcolor[rgb]{0.13,0.29,0.53}{#1}}
\newcommand{\DecValTok}[1]{\textcolor[rgb]{0.00,0.00,0.81}{#1}}
\newcommand{\BaseNTok}[1]{\textcolor[rgb]{0.00,0.00,0.81}{#1}}
\newcommand{\FloatTok}[1]{\textcolor[rgb]{0.00,0.00,0.81}{#1}}
\newcommand{\ConstantTok}[1]{\textcolor[rgb]{0.00,0.00,0.00}{#1}}
\newcommand{\CharTok}[1]{\textcolor[rgb]{0.31,0.60,0.02}{#1}}
\newcommand{\SpecialCharTok}[1]{\textcolor[rgb]{0.00,0.00,0.00}{#1}}
\newcommand{\StringTok}[1]{\textcolor[rgb]{0.31,0.60,0.02}{#1}}
\newcommand{\VerbatimStringTok}[1]{\textcolor[rgb]{0.31,0.60,0.02}{#1}}
\newcommand{\SpecialStringTok}[1]{\textcolor[rgb]{0.31,0.60,0.02}{#1}}
\newcommand{\ImportTok}[1]{#1}
\newcommand{\CommentTok}[1]{\textcolor[rgb]{0.56,0.35,0.01}{\textit{#1}}}
\newcommand{\DocumentationTok}[1]{\textcolor[rgb]{0.56,0.35,0.01}{\textbf{\textit{#1}}}}
\newcommand{\AnnotationTok}[1]{\textcolor[rgb]{0.56,0.35,0.01}{\textbf{\textit{#1}}}}
\newcommand{\CommentVarTok}[1]{\textcolor[rgb]{0.56,0.35,0.01}{\textbf{\textit{#1}}}}
\newcommand{\OtherTok}[1]{\textcolor[rgb]{0.56,0.35,0.01}{#1}}
\newcommand{\FunctionTok}[1]{\textcolor[rgb]{0.00,0.00,0.00}{#1}}
\newcommand{\VariableTok}[1]{\textcolor[rgb]{0.00,0.00,0.00}{#1}}
\newcommand{\ControlFlowTok}[1]{\textcolor[rgb]{0.13,0.29,0.53}{\textbf{#1}}}
\newcommand{\OperatorTok}[1]{\textcolor[rgb]{0.81,0.36,0.00}{\textbf{#1}}}
\newcommand{\BuiltInTok}[1]{#1}
\newcommand{\ExtensionTok}[1]{#1}
\newcommand{\PreprocessorTok}[1]{\textcolor[rgb]{0.56,0.35,0.01}{\textit{#1}}}
\newcommand{\AttributeTok}[1]{\textcolor[rgb]{0.77,0.63,0.00}{#1}}
\newcommand{\RegionMarkerTok}[1]{#1}
\newcommand{\InformationTok}[1]{\textcolor[rgb]{0.56,0.35,0.01}{\textbf{\textit{#1}}}}
\newcommand{\WarningTok}[1]{\textcolor[rgb]{0.56,0.35,0.01}{\textbf{\textit{#1}}}}
\newcommand{\AlertTok}[1]{\textcolor[rgb]{0.94,0.16,0.16}{#1}}
\newcommand{\ErrorTok}[1]{\textcolor[rgb]{0.64,0.00,0.00}{\textbf{#1}}}
\newcommand{\NormalTok}[1]{#1}
\usepackage{graphicx,grffile}
\makeatletter
\def\maxwidth{\ifdim\Gin@nat@width>\linewidth\linewidth\else\Gin@nat@width\fi}
\def\maxheight{\ifdim\Gin@nat@height>\textheight\textheight\else\Gin@nat@height\fi}
\makeatother
% Scale images if necessary, so that they will not overflow the page
% margins by default, and it is still possible to overwrite the defaults
% using explicit options in \includegraphics[width, height, ...]{}
\setkeys{Gin}{width=\maxwidth,height=\maxheight,keepaspectratio}
\IfFileExists{parskip.sty}{%
\usepackage{parskip}
}{% else
\setlength{\parindent}{0pt}
\setlength{\parskip}{6pt plus 2pt minus 1pt}
}
\setlength{\emergencystretch}{3em}  % prevent overfull lines
\providecommand{\tightlist}{%
  \setlength{\itemsep}{0pt}\setlength{\parskip}{0pt}}
\setcounter{secnumdepth}{0}
% Redefines (sub)paragraphs to behave more like sections
\ifx\paragraph\undefined\else
\let\oldparagraph\paragraph
\renewcommand{\paragraph}[1]{\oldparagraph{#1}\mbox{}}
\fi
\ifx\subparagraph\undefined\else
\let\oldsubparagraph\subparagraph
\renewcommand{\subparagraph}[1]{\oldsubparagraph{#1}\mbox{}}
\fi

%%% Use protect on footnotes to avoid problems with footnotes in titles
\let\rmarkdownfootnote\footnote%
\def\footnote{\protect\rmarkdownfootnote}

%%% Change title format to be more compact
\usepackage{titling}

% Create subtitle command for use in maketitle
\newcommand{\subtitle}[1]{
  \posttitle{
    \begin{center}\large#1\end{center}
    }
}

\setlength{\droptitle}{-2em}

  \title{Capitulo 2. Análisis de Sentimientos}
    \pretitle{\vspace{\droptitle}\centering\huge}
  \posttitle{\par}
    \author{}
    \preauthor{}\postauthor{}
    \date{}
    \predate{}\postdate{}
  

\begin{document}
\maketitle

This is an \href{http://rmarkdown.rstudio.com}{R Markdown} Notebook.
When you execute code within the notebook, the results appear beneath
the code.

Try executing this chunk by clicking the \emph{Run} button within the
chunk or by placing your cursor inside it and pressing
\emph{Ctrl+Shift+Enter}.

\begin{Shaded}
\begin{Highlighting}[]
\KeywordTok{library}\NormalTok{(tidytext)}

\NormalTok{sentiments}
\end{Highlighting}
\end{Shaded}

\begin{verbatim}
## # A tibble: 27,314 x 4
##    word        sentiment lexicon score
##    <chr>       <chr>     <chr>   <int>
##  1 abacus      trust     nrc        NA
##  2 abandon     fear      nrc        NA
##  3 abandon     negative  nrc        NA
##  4 abandon     sadness   nrc        NA
##  5 abandoned   anger     nrc        NA
##  6 abandoned   fear      nrc        NA
##  7 abandoned   negative  nrc        NA
##  8 abandoned   sadness   nrc        NA
##  9 abandonment anger     nrc        NA
## 10 abandonment fear      nrc        NA
## # ... with 27,304 more rows
\end{verbatim}

Add a new chunk by clicking the \emph{Insert Chunk} button on the
toolbar or by pressing \emph{Ctrl+Alt+I}.

When you save the notebook, an HTML file containing the code and output
will be saved alongside it (click the \emph{Preview} button or press
\emph{Ctrl+Shift+K} to preview the HTML file).

The preview shows you a rendered HTML copy of the contents of the
editor. Consequently, unlike \emph{Knit}, \emph{Preview} does not run
any R code chunks. Instead, the output of the chunk when it was last run
in the editor is displayed.

\begin{Shaded}
\begin{Highlighting}[]
\KeywordTok{get_sentiments}\NormalTok{(}\StringTok{"afinn"}\NormalTok{)}
\end{Highlighting}
\end{Shaded}

\begin{verbatim}
## # A tibble: 2,476 x 2
##    word       score
##    <chr>      <int>
##  1 abandon       -2
##  2 abandoned     -2
##  3 abandons      -2
##  4 abducted      -2
##  5 abduction     -2
##  6 abductions    -2
##  7 abhor         -3
##  8 abhorred      -3
##  9 abhorrent     -3
## 10 abhors        -3
## # ... with 2,466 more rows
\end{verbatim}

\begin{Shaded}
\begin{Highlighting}[]
\KeywordTok{get_sentiments}\NormalTok{(}\StringTok{"bing"}\NormalTok{)}
\end{Highlighting}
\end{Shaded}

\begin{verbatim}
## # A tibble: 6,788 x 2
##    word        sentiment
##    <chr>       <chr>    
##  1 2-faced     negative 
##  2 2-faces     negative 
##  3 a+          positive 
##  4 abnormal    negative 
##  5 abolish     negative 
##  6 abominable  negative 
##  7 abominably  negative 
##  8 abominate   negative 
##  9 abomination negative 
## 10 abort       negative 
## # ... with 6,778 more rows
\end{verbatim}

\begin{Shaded}
\begin{Highlighting}[]
\KeywordTok{get_sentiments}\NormalTok{(}\StringTok{"nrc"}\NormalTok{)}
\end{Highlighting}
\end{Shaded}

\begin{verbatim}
## # A tibble: 13,901 x 2
##    word        sentiment
##    <chr>       <chr>    
##  1 abacus      trust    
##  2 abandon     fear     
##  3 abandon     negative 
##  4 abandon     sadness  
##  5 abandoned   anger    
##  6 abandoned   fear     
##  7 abandoned   negative 
##  8 abandoned   sadness  
##  9 abandonment anger    
## 10 abandonment fear     
## # ... with 13,891 more rows
\end{verbatim}

\begin{Shaded}
\begin{Highlighting}[]
\KeywordTok{library}\NormalTok{(janeaustenr)}
\KeywordTok{library}\NormalTok{(dplyr)}
\end{Highlighting}
\end{Shaded}

\begin{verbatim}
## 
## Attaching package: 'dplyr'
\end{verbatim}

\begin{verbatim}
## The following objects are masked from 'package:stats':
## 
##     filter, lag
\end{verbatim}

\begin{verbatim}
## The following objects are masked from 'package:base':
## 
##     intersect, setdiff, setequal, union
\end{verbatim}

\begin{Shaded}
\begin{Highlighting}[]
\KeywordTok{library}\NormalTok{(stringr)}

\NormalTok{tidy_books <-}\StringTok{ }\KeywordTok{austen_books}\NormalTok{() }\OperatorTok
\StringTok{  }\KeywordTok{group_by}\NormalTok{(book) }\OperatorTok
\StringTok{  }\KeywordTok{mutate}\NormalTok{(}\DataTypeTok{linenumber =} \KeywordTok{row_number}\NormalTok{(),}
         \DataTypeTok{chapter =} \KeywordTok{cumsum}\NormalTok{(}\KeywordTok{str_detect}\NormalTok{(text, }\KeywordTok{regex}\NormalTok{(}\StringTok{"^chapter [}\CharTok{\textbackslash{}\textbackslash{}}\StringTok{divxlc]"}\NormalTok{, }
                                                 \DataTypeTok{ignore_case =} \OtherTok{TRUE}\NormalTok{)))) }\OperatorTok
\StringTok{  }\KeywordTok{ungroup}\NormalTok{() }\OperatorTok
\StringTok{  }\KeywordTok{unnest_tokens}\NormalTok{(word, text)}

\NormalTok{tidy_books}
\end{Highlighting}
\end{Shaded}

\begin{verbatim}
## # A tibble: 725,055 x 4
##    book                linenumber chapter word       
##    <fct>                    <int>   <int> <chr>      
##  1 Sense & Sensibility          1       0 sense      
##  2 Sense & Sensibility          1       0 and        
##  3 Sense & Sensibility          1       0 sensibility
##  4 Sense & Sensibility          3       0 by         
##  5 Sense & Sensibility          3       0 jane       
##  6 Sense & Sensibility          3       0 austen     
##  7 Sense & Sensibility          5       0 1811       
##  8 Sense & Sensibility         10       1 chapter    
##  9 Sense & Sensibility         10       1 1          
## 10 Sense & Sensibility         13       1 the        
## # ... with 725,045 more rows
\end{verbatim}

\begin{Shaded}
\begin{Highlighting}[]
\NormalTok{nrc_joy <-}\StringTok{ }\KeywordTok{get_sentiments}\NormalTok{(}\StringTok{"nrc"}\NormalTok{) }\OperatorTok\StringTok{ }
\StringTok{  }\KeywordTok{filter}\NormalTok{(sentiment }\OperatorTok{==}\StringTok{ "joy"}\NormalTok{)}

\NormalTok{tidy_books }\OperatorTok
\StringTok{  }\KeywordTok{filter}\NormalTok{(book }\OperatorTok{==}\StringTok{ "Emma"}\NormalTok{) }\OperatorTok
\StringTok{  }\KeywordTok{inner_join}\NormalTok{(nrc_joy) }\OperatorTok
\StringTok{  }\KeywordTok{count}\NormalTok{(word, }\DataTypeTok{sort =} \OtherTok{TRUE}\NormalTok{)}
\end{Highlighting}
\end{Shaded}

\begin{verbatim}
## Joining, by = "word"
\end{verbatim}

\begin{verbatim}
## # A tibble: 303 x 2
##    word        n
##    <chr>   <int>
##  1 good      359
##  2 young     192
##  3 friend    166
##  4 hope      143
##  5 happy     125
##  6 love      117
##  7 deal       92
##  8 found      92
##  9 present    89
## 10 kind       82
## # ... with 293 more rows
\end{verbatim}

\begin{Shaded}
\begin{Highlighting}[]
\KeywordTok{library}\NormalTok{(tidyr)}

\NormalTok{jane_austen_sentiment <-}\StringTok{ }\NormalTok{tidy_books }\OperatorTok
\StringTok{  }\KeywordTok{inner_join}\NormalTok{(}\KeywordTok{get_sentiments}\NormalTok{(}\StringTok{"bing"}\NormalTok{)) }\OperatorTok
\StringTok{  }\KeywordTok{count}\NormalTok{(book, }\DataTypeTok{index =}\NormalTok{ linenumber }\OperatorTok\StringTok{ }\DecValTok{80}\NormalTok{, sentiment) }\OperatorTok
\StringTok{  }\KeywordTok{spread}\NormalTok{(sentiment, n, }\DataTypeTok{fill =} \DecValTok{0}\NormalTok{) }\OperatorTok
\StringTok{  }\KeywordTok{mutate}\NormalTok{(}\DataTypeTok{sentiment =}\NormalTok{ positive }\OperatorTok{-}\StringTok{ }\NormalTok{negative)}
\end{Highlighting}
\end{Shaded}

\begin{verbatim}
## Joining, by = "word"
\end{verbatim}

\begin{Shaded}
\begin{Highlighting}[]
\NormalTok{jane_austen_sentiment}
\end{Highlighting}
\end{Shaded}

\begin{verbatim}
## # A tibble: 920 x 5
##    book                index negative positive sentiment
##    <fct>               <dbl>    <dbl>    <dbl>     <dbl>
##  1 Sense & Sensibility     0       16       32        16
##  2 Sense & Sensibility     1       19       53        34
##  3 Sense & Sensibility     2       12       31        19
##  4 Sense & Sensibility     3       15       31        16
##  5 Sense & Sensibility     4       16       34        18
##  6 Sense & Sensibility     5       16       51        35
##  7 Sense & Sensibility     6       24       40        16
##  8 Sense & Sensibility     7       23       51        28
##  9 Sense & Sensibility     8       30       40        10
## 10 Sense & Sensibility     9       15       19         4
## # ... with 910 more rows
\end{verbatim}

\begin{Shaded}
\begin{Highlighting}[]
\KeywordTok{library}\NormalTok{(ggplot2)}

\KeywordTok{ggplot}\NormalTok{(jane_austen_sentiment, }\KeywordTok{aes}\NormalTok{(index, sentiment, }\DataTypeTok{fill =}\NormalTok{ book)) }\OperatorTok{+}
\StringTok{  }\KeywordTok{geom_col}\NormalTok{(}\DataTypeTok{show.legend =} \OtherTok{FALSE}\NormalTok{) }\OperatorTok{+}
\StringTok{  }\KeywordTok{facet_wrap}\NormalTok{(}\OperatorTok{~}\NormalTok{book, }\DataTypeTok{ncol =} \DecValTok{2}\NormalTok{, }\DataTypeTok{scales =} \StringTok{"free_x"}\NormalTok{)}
\end{Highlighting}
\end{Shaded}

\includegraphics{tx_5_files/figure-latex/unnamed-chunk-8-1.pdf}

\begin{Shaded}
\begin{Highlighting}[]
\NormalTok{pride_prejudice <-}\StringTok{ }\NormalTok{tidy_books }\OperatorTok\StringTok{ }
\StringTok{  }\KeywordTok{filter}\NormalTok{(book }\OperatorTok{==}\StringTok{ "Pride & Prejudice"}\NormalTok{)}

\NormalTok{pride_prejudice}
\end{Highlighting}
\end{Shaded}

\begin{verbatim}
## # A tibble: 122,204 x 4
##    book              linenumber chapter word     
##    <fct>                  <int>   <int> <chr>    
##  1 Pride & Prejudice          1       0 pride    
##  2 Pride & Prejudice          1       0 and      
##  3 Pride & Prejudice          1       0 prejudice
##  4 Pride & Prejudice          3       0 by       
##  5 Pride & Prejudice          3       0 jane     
##  6 Pride & Prejudice          3       0 austen   
##  7 Pride & Prejudice          7       1 chapter  
##  8 Pride & Prejudice          7       1 1        
##  9 Pride & Prejudice         10       1 it       
## 10 Pride & Prejudice         10       1 is       
## # ... with 122,194 more rows
\end{verbatim}

\begin{Shaded}
\begin{Highlighting}[]
\NormalTok{afinn <-}\StringTok{ }\NormalTok{pride_prejudice }\OperatorTok\StringTok{ }
\StringTok{  }\KeywordTok{inner_join}\NormalTok{(}\KeywordTok{get_sentiments}\NormalTok{(}\StringTok{"afinn"}\NormalTok{)) }\OperatorTok\StringTok{ }
\StringTok{  }\KeywordTok{group_by}\NormalTok{(}\DataTypeTok{index =}\NormalTok{ linenumber }\OperatorTok\StringTok{ }\DecValTok{80}\NormalTok{) }\OperatorTok\StringTok{ }
\StringTok{  }\KeywordTok{summarise}\NormalTok{(}\DataTypeTok{sentiment =} \KeywordTok{sum}\NormalTok{(score)) }\OperatorTok\StringTok{ }
\StringTok{  }\KeywordTok{mutate}\NormalTok{(}\DataTypeTok{method =} \StringTok{"AFINN"}\NormalTok{)}
\end{Highlighting}
\end{Shaded}

\begin{verbatim}
## Joining, by = "word"
\end{verbatim}

\begin{Shaded}
\begin{Highlighting}[]
\NormalTok{afinn}
\end{Highlighting}
\end{Shaded}

\begin{verbatim}
## # A tibble: 163 x 3
##    index sentiment method
##    <dbl>     <int> <chr> 
##  1     0        29 AFINN 
##  2     1         0 AFINN 
##  3     2        20 AFINN 
##  4     3        30 AFINN 
##  5     4        62 AFINN 
##  6     5        66 AFINN 
##  7     6        60 AFINN 
##  8     7        18 AFINN 
##  9     8        84 AFINN 
## 10     9        26 AFINN 
## # ... with 153 more rows
\end{verbatim}

\begin{Shaded}
\begin{Highlighting}[]
\NormalTok{bing_and_nrc <-}\StringTok{ }\KeywordTok{bind_rows}\NormalTok{(pride_prejudice }\OperatorTok\StringTok{ }
\StringTok{                            }\KeywordTok{inner_join}\NormalTok{(}\KeywordTok{get_sentiments}\NormalTok{(}\StringTok{"bing"}\NormalTok{)) }\OperatorTok
\StringTok{                            }\KeywordTok{mutate}\NormalTok{(}\DataTypeTok{method =} \StringTok{"Bing et al."}\NormalTok{),}
\NormalTok{                          pride_prejudice }\OperatorTok\StringTok{ }
\StringTok{                            }\KeywordTok{inner_join}\NormalTok{(}\KeywordTok{get_sentiments}\NormalTok{(}\StringTok{"nrc"}\NormalTok{) }\OperatorTok\StringTok{ }
\StringTok{                                         }\KeywordTok{filter}\NormalTok{(sentiment }\OperatorTok\StringTok{ }\KeywordTok{c}\NormalTok{(}\StringTok{"positive"}\NormalTok{, }
                                                                 \StringTok{"negative"}\NormalTok{))) }\OperatorTok
\StringTok{                            }\KeywordTok{mutate}\NormalTok{(}\DataTypeTok{method =} \StringTok{"NRC"}\NormalTok{)) }\OperatorTok
\StringTok{  }\KeywordTok{count}\NormalTok{(method, }\DataTypeTok{index =}\NormalTok{ linenumber }\OperatorTok\StringTok{ }\DecValTok{80}\NormalTok{, sentiment) }\OperatorTok
\StringTok{  }\KeywordTok{spread}\NormalTok{(sentiment, n, }\DataTypeTok{fill =} \DecValTok{0}\NormalTok{) }\OperatorTok
\StringTok{  }\KeywordTok{mutate}\NormalTok{(}\DataTypeTok{sentiment =}\NormalTok{ positive }\OperatorTok{-}\StringTok{ }\NormalTok{negative)}
\end{Highlighting}
\end{Shaded}

\begin{verbatim}
## Joining, by = "word"
## Joining, by = "word"
\end{verbatim}

\begin{Shaded}
\begin{Highlighting}[]
\NormalTok{bing_and_nrc}
\end{Highlighting}
\end{Shaded}

\begin{verbatim}
## # A tibble: 326 x 5
##    method      index negative positive sentiment
##    <chr>       <dbl>    <dbl>    <dbl>     <dbl>
##  1 Bing et al.     0        7       21        14
##  2 Bing et al.     1       20       19        -1
##  3 Bing et al.     2       16       20         4
##  4 Bing et al.     3       19       31        12
##  5 Bing et al.     4       23       47        24
##  6 Bing et al.     5       15       49        34
##  7 Bing et al.     6       18       46        28
##  8 Bing et al.     7       23       33        10
##  9 Bing et al.     8       17       48        31
## 10 Bing et al.     9       22       40        18
## # ... with 316 more rows
\end{verbatim}

\begin{Shaded}
\begin{Highlighting}[]
\KeywordTok{bind_rows}\NormalTok{(afinn, }
\NormalTok{          bing_and_nrc) }\OperatorTok
\StringTok{  }\KeywordTok{ggplot}\NormalTok{(}\KeywordTok{aes}\NormalTok{(index, sentiment, }\DataTypeTok{fill =}\NormalTok{ method)) }\OperatorTok{+}
\StringTok{  }\KeywordTok{geom_col}\NormalTok{(}\DataTypeTok{show.legend =} \OtherTok{FALSE}\NormalTok{) }\OperatorTok{+}
\StringTok{  }\KeywordTok{facet_wrap}\NormalTok{(}\OperatorTok{~}\NormalTok{method, }\DataTypeTok{ncol =} \DecValTok{1}\NormalTok{, }\DataTypeTok{scales =} \StringTok{"free_y"}\NormalTok{)}
\end{Highlighting}
\end{Shaded}

\includegraphics{tx_5_files/figure-latex/unnamed-chunk-12-1.pdf}


\end{document}
